% !TEX TS-program = pdflatex
% !TEX encoding = UTF-8 Unicode

% This is a simple template for a LaTeX document using the "article" class.
% See "book", "report", "letter" for other types of document.

\documentclass[11pt]{article} % use larger type; default would be 10pt

\usepackage[utf8]{inputenc} % set input encoding (not needed with XeLaTeX)

%%% Examples of Article customizations
% These packages are optional, depending whether you want the features they provide.
% See the LaTeX Companion or other references for full information.

%%% PAGE DIMENSIONS
\usepackage{geometry} % to change the page dimensions
\usepackage{amsfonts}
\usepackage{color}
\geometry{a4paper} % or letterpaper (US) or a5paper or....
% \geometry{margins=2in} % for example, change the margins to 2 inches all round
% \geometry{landscape} % set up the page for landscape
%   read geometry.pdf for detailed page layout information

\usepackage{graphicx} % support the \includegraphics command and options

% \usepackage[parfill]{parskip} % Activate to begin paragraphs with an empty line rather than an indent

%%% PACKAGES
\usepackage{booktabs} % for much better looking tables
\usepackage{array} % for better arrays (eg matrices) in maths
\usepackage{paralist} % very flexible & customisable lists (eg. enumerate/itemize, etc.)
\usepackage{verbatim} % adds environment for commenting out blocks of text & for better verbatim
\usepackage{subfig} % make it possible to include more than one captioned figure/table in a single float
% These packages are all incorporated in the memoir class to one degree or another...

%%% HEADERS & FOOTERS
\usepackage{fancyhdr} % This should be set AFTER setting up the page geometry
\pagestyle{fancy} % options: empty , plain , fancy
\renewcommand{\headrulewidth}{0pt} % customise the layout...
\lhead{}\chead{}\rhead{}
\lfoot{}\cfoot{\thepage}\rfoot{}

%%% SECTION TITLE APPEARANCE
\usepackage{sectsty}
\allsectionsfont{\sffamily\mdseries\upshape} % (See the fntguide.pdf for font help)
% (This matches ConTeXt defaults)

%%% ToC (table of contents) APPEARANCE
\usepackage[nottoc,notlof,notlot]{tocbibind} % Put the bibliography in the ToC
\usepackage[titles,subfigure]{tocloft} % Alter the style of the Table of Contents
\renewcommand{\cftsecfont}{\rmfamily\mdseries\upshape}
\renewcommand{\cftsecpagefont}{\rmfamily\mdseries\upshape} % No bold!
\usepackage{amsmath}
%%% END Article customizations

\setlength{\parindent}{0in}

%%% The "real" document content comes below...



\title{A 405 Day 12 Problems}

\author{Niamh Chaparro}
%\date{} % Activate to display a given date or no date (if empty),
         % otherwise the current date is printed 

\begin{document}

\maketitle

1\\

The vapour pressure at V1 is $e = \frac{RH \times e_{sat}}{100} = .6 \times 2.337 \times 10^{3}  = 1402.2 Pa$\\

e at V1 can be used to obtain the water vapour density at V1\\

$\rho_v^{'} = \frac{e}{R_{v}T} = \frac{1402.2}{461.51 \times 293.15} = .010364 \ kg \ m^{-3}$\\

and the mass is obtained as follows: $m_{w} = .010364 \times 20 \times 10^{-3} = .00020728 \ kg$.\\

Does the change in volume result in vapour pressure exceeding saturation?\\

$e_{2} = \frac{V_{1}}{V_{2}}e_{1} = \frac{20}{4}\times 1402.2 = 7011 \ Pa > e_{sat}$\\

Yes, so the vapour pressure at V2 is $e_{sat}$ and the vapour density at V2 is\\

$ \rho_{v2}^{'} = \frac{e_{sat}}{R_{v}T} = \frac{2337}{461.51 \times 293.15} = .017273792 \ kg \ m^{-3}$\\

The new vapour mass is $.017273793 \times 4 \times 10^{-3} = 6.9095 \times 10^{-5} \ kg$\\

so the mass of condensed water is  $(2.0728 - .69095) \times 10^{-4} = 1.38185 \times 10^{-4} \ kg$.\\


2\\

a\\

The potential temperature is conserved: $\frac{\theta_{1}}{\theta_{2}} = 1 = \frac{T_{1}}{T_{2}}(\frac{P_{0}P_{2}}{P_{0}P_{1}})^{.286}$\\

$T_{2} = T_{1}(\frac{P_{2}}{P_{1}})^{.286} = 373.783 K$\\

(Aside:\\

For the lower compartment $V_{2} = (1-X)V_{1}$ so $(1-X) = \frac{373.783 \times 1}{273 \times 3}$ and the difference in volume $X = 1 - \frac{373.783}{273 \times 3} = .544$
)\\

b\\

For the lower compartment $\Delta q = \Delta u - \Delta w$.  Since there is no thermal energy supplied or lost $\Delta q = 0$ and the change in internal energy is equal to the work done by the thermally induced pressure increase in the upper compartment via the energetically neutral membrane: $\Delta u = \Delta w = c_{v} \Delta T = 72261.276 \ J \ kg^{-1}$\\

c\\

$\frac{T_{1}}{T_{2}} = \frac{P_{1}V_{1}}{P_{2}V_{2}} = \frac{1}{3 \times (1 + X)}$\\

$T_{2} = 3 \times (1 + X) \times T_{1} = 1192 \ K$\\

d\\

The total thermal energy supplied in the upper compartment is that which causes the temperature to increase and the pressure and volume to increase:\\

$\Delta q = \Delta u + \Delta w = c_{v} \times \Delta T + \Delta w = 717 \times (1192 - 273) + 72261.276 = 731184.276 \ J \ kg ^{-1}$\\

\end{document}