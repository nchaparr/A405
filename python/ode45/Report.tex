% !TEX TS-program = pdflatex
% !TEX encoding = UTF-8 Unicode

% This is a simple template for a LaTeX document using the "article" class.
% See "book", "report", "letter" for other types of document.

\documentclass[11pt]{article} % use larger type; default would be 10pt

\usepackage[utf8]{inputenc} % set input encoding (not needed with XeLaTeX)

%%% Examples of Article customizations
% These packages are optional, depending whether you want the features they provide.
% See the LaTeX Companion or other references for full information.

%%% PAGE DIMENSIONS
\usepackage{geometry} % to change the page dimensions
\usepackage{amsfonts}
\usepackage{color}
\geometry{a4paper} % or letterpaper (US) or a5paper or....
% \geometry{margins=2in} % for example, change the margins to 2 inches all round
% \geometry{landscape} % set up the page for landscape
%   read geometry.pdf for detailed page layout information

\usepackage{graphicx} % support the \includegraphics command and options

% \usepackage[parfill]{parskip} % Activate to begin paragraphs with an empty line rather than an indent

%%% PACKAGES
\usepackage{booktabs} % for much better looking tables
\usepackage{array} % for better arrays (eg matrices) in maths
\usepackage{paralist} % very flexible & customisable lists (eg. enumerate/itemize, etc.)
\usepackage{verbatim} % adds environment for commenting out blocks of text & for better verbatim
\usepackage{subfig} % make it possible to include more than one captioned figure/table in a single float
% These packages are all incorporated in the memoir class to one degree or another...

%%% HEADERS & FOOTERS
\usepackage{fancyhdr} % This should be set AFTER setting up the page geometry
\pagestyle{fancy} % options: empty , plain , fancy
\renewcommand{\headrulewidth}{0pt} % customise the layout...
\lhead{}\chead{}\rhead{}
\lfoot{}\cfoot{\thepage}\rfoot{}

%%% SECTION TITLE APPEARANCE
\usepackage{sectsty}
\allsectionsfont{\sffamily\mdseries\upshape} % (See the fntguide.pdf for font help)
% (This matches ConTeXt defaults)

%%% ToC (table of contents) APPEARANCE
\usepackage[nottoc,notlof,notlot]{tocbibind} % Put the bibliography in the ToC
\usepackage[titles,subfigure]{tocloft} % Alter the style of the Table of Contents
\renewcommand{\cftsecfont}{\rmfamily\mdseries\upshape}
\renewcommand{\cftsecpagefont}{\rmfamily\mdseries\upshape} % No bold!
\usepackage{amsmath}
%%% END Article customizations

\setlength{\parindent}{0in}

%%% The "real" document content comes below...

\title{Cloud Physics Course Project}

\author{Niamh Chaparro}
%\date{} % Activate to display a given date or no date (if empty),
         % otherwise the current date is printed 

\begin{document}

\maketitle

\section{Background}

\subsection{buoyancy}

Buoyant convection leads to the formation of cumulous clouds.  A parcel of dry air, less dense that its surroundings will have an associated buoyancy force (vertical aceleration).\\

$\frac{d^{2}z}{dt^{2}} = gB, B = \frac{\rho^{'} - \rho}{rho} = \frac{T - T^{'}}{T^{'}}, ' denotes surroundings$\\

The upward motion will further be influenced by water loading, friction and mixing with different air.  For moist air virtual temperature is used\\.  

$T_{v} \approx T[1 + 0.6w], w  = \frac{M_{v}}{M_{d}}$\\

There's also a version which takes liquid water into account.\\  

We ignore the other influencing factors mentioned above, so assume no mixing occurs and the parcel follows a pseudoadiabat{\bf verify with Phil about the inherent assumption that liquid falls out of the parcel mentioned in Rogers and Yau}.  As the parcel rises due to buoyancy it adjusts to the surrounding pressure, so it's temperature will change.  Thus saturation water mixing ratio will change leading to supersaturation and condensation (assuming an already saturated parcel, and from a macroscopic perspective).  A simple and macroscale persective is to say that once saturation is reached, condensation occurs such that the water vapour mixing ration does not exceed the saturation value.  It is also possible to view condensation from the perspective of droplet formation.

\subsection{Rate of change of Temperature with Time} 

If we assume pseudoadiabatic lifting of a saturated parcel, the equivalent potential temperature:\\

$\theta_{e} = T \times (\frac{P_{0}}{P_{d}})^{\frac{R_{d}}{C_{p}}}\times exp(\frac{L_{v}w_{sat}}{C_{p}T})$\\

is conserved with respect to time, so $\frac{d \theta_{e}}{dt} = $\\

$\frac{dT}{dt}(\frac{P_{0}}{P_{d}})^{\frac{R_{d}}{C_{p}}}  - \frac{R_{d}T}{C_{p}}(\frac{P_{0}}{P_{d}})^{\frac{R_{d}}{C_{p}}}\frac{1}{P_{d}}\frac{dP_{d}}{dt} + \frac{d}{dt} (\frac{L_{v}w_{s}}{C_{p}T}) \times T(\frac{P_{0}}{P_{d}})^{\frac{R_{d}}{C_{p}}} = 0$\\

$\frac{dT}{dt}  - \frac{R_{d}T}{C_{p}}\frac{1}{P_{d}}\frac{dP_{d}}{dt} - \frac{L_{v}w_{s}}{C_{p}T}\frac{dT}{dt} + \frac{L_{v}}{C_{p}}\frac{dT}{dt}\frac{dw_{s}}{dT}+ \frac{L_{v}}{C_{p}}\frac{dw_{s}}{dP_{d}}\frac{dP_{d}}{dt} = 0$\\

$\frac{dT}{dt}(1 - \frac{L_{v}w_{s}}{C_{p}T} + \frac{L_{v}}{C_{p}}\frac{dw_{s}}{dT}) = (\frac{R_{d}T}{C_{p}P_{d}} - \frac{L_{v}}{C_{p}}\frac{dw_{s}}{dP})\frac{dP_{d}}{dt} = 0$\\

Assuming hydrostatic balance

$\frac{dT}{dt} = (\frac{R_{d}T}{C_{p}P_{d}} - \frac{L_{v}}{C_{p}}\frac{dw_{s}}{dP})(1 - \frac{L_{v}w_{s}}{C_{p}T} + \frac{L_{v}}{C_{p}}\frac{dw_{s}}{dT})^{-1} \times w \times (-\rho g)$\\

The following draws upon the Calusius-Claperon equation

$\frac{dw_{s}}{dT} = (w_{s} + w_{s}^{2})\frac{1}{e_{s}}\frac{de_{s}}{dT}, \frac{de_{s}}{dT} = \frac{L_{v} e_{s}}{R_{v}T^{2}}, \frac{dw_{s}}{dP} = -\frac{\epsilon \times e_{s}}{(P - e_{s})^{2}}$ \\

\subsection{Cloud Droplet Nucleation}

A system with liquid and vapour, by definition is at saturation when the rates of evapouration and condensation are equal.  This does occur for example over a large mass of water.  For air borne pure water vapour to condense in droplets, a significant free energy barrier must be overcome.  Molecules must collide and agreggate.  Surface tension dictates that a droplet will be sperical, and must be overcome to expand its surface. $dW_{s} = \sigma dA $ So energy is consumed by surface tension and released in the form of the latent heat of condensation.  Also growth is dependent on the surrounding vapor pressure, ie the number of water molecules impinging on its surface.  Evaporation then depends on whether the molecules in the droplet have enough energy to overcome the intermolecular bonds.  Equilibrium is reached when the vapor pressure next to the surface of the droplet is saturated. The saturation vapour pressure for the droplet is obtained from that over bulk water: $e_{s}(r) = e_{s}(\infty) exp(\frac{2 \sigma}{r R_{v}rho_{L} T})$  The overall growth so is related (proportional to) to the difference between the ambient vapor pressure and the saturation vapor pressure of the droplet. $e - e_{s}(r)$ When this is positive (ie an available excess of water molecules) the droplet will tente to grow.  The critical radius $\frac{2\sigma}{R_{v}\rho_{L}T ln(\frac{e_{s}(r)}{e_{s}(\infty)})}$ of the droplet is obtained by setting this term equal to zero.  The droplet must attain this value before it is stable.  The presence of hydroscopic aerosols greatly increase rates of droplet formation.  The equilibrium vapour pressure of over a plane surface of non-volatile aerosol dissolved in water will be lower per raoult's law $\frac{e^{'}_{s}}{e_{s}(\infty)} = \frac{n_{0}}{n + n_{0}}\ n_{0} = 1 - \frac{n}{n_{0}} number \ of \ water \ molecules, n \ = \ number \ of \ dissolved \ solute \ molecules$ Note that $n$  is $i$ or the Van't Hoff factor multiplied by the number of dry aerosol molecules. So $n = iN_{A}\frac{m_{s}}{M_{s}}, n_{0} = \frac{N_{0}m_{w}}{M_{w}}, m_{w} = \frac{4}{3} \pi r^{3} \rho_{L}$ and $\frac{e^{'}_{s}(r)}{e_{s}(\infty)} = [1 - \frac{b}{r^{3}}]e^{\frac{a}{r}}, a = \frac{2 \sigma}{r R_{v} \rho_{L}T}, b = \frac{3iM_{v} m_{s}}{4 \pi \rho_{L} M_{s}}$  Using the first order taylor series approximation for the exponential term, and ignoring the product of the two small terms gives $\frac{e^{'}_{s}(r)}{e_{s}(\infty)} = 1 + \frac{a}{r} - \frac{b}{r^{3}}$  (need to have plot or something for kholer curve.  Note that my script assumes $e^{'}_{s}(r) = e(\infty)$  ) The maximimum on the resulting curve represents the critical radius and value  of $\frac{e^{'}_{s}(r)}{e_{s}(\infty)}$, $r^{*} = \sqrt{\frac{3b}{a}}, S^{*} = 1 + \sqrt{\frac{4a^{3}}{27b}}$

\subsection{Diffusional Droplet Growth}

The droplet grows by diffusion of water to its surface.  Consider a droplet with a number of water molecules a distance $R$ from its centre, $n(R)$.  Assume isotropy (uniformaty in all directions) and steady state so $\frac{\partial n}{\partial t} = D \nabla^{2} n = 0$  Converting to spherical coordinates, where only change in the radial direction is considered $\nabla^{2}n(R) = \frac{1}{R^{2}}\frac{\partial}{\partial R} \left( R^{2} \frac{\partial n}{\partial R} \right) = 0$ Solving yields: $n(R) = C_{1} - \frac{C_{2}}{R}$ where $R \rightarrow \infty, n \rightarrow n_{\infty}$ and $R \rightarrow r, n \rightarrow n_{r}$ Using these boundary conditions gives $n(R) = n_{\infty} - \frac{r}{R}(n_{\infty} - n_{r})$ Now the flux of molecules at the surface of the droplet is $D \left( \frac{\partial n}{\partial R} \right)_{R = r}$. So using the differential of $n(R)$  at $R = r$ the rate of mass increase (using density instead of number of molecules): $\frac{dm}{dt} = 4 \pi r D (\rho_{\infty} - \rho_{r})$.  Substition of $m = \frac{4}{3}\pi r^{3}$ into the right hand side yields $\frac{dr}{dt} = \frac{D\rho_{v}(\infty)}{r\rho_{l}e(\infty)[e(\infty) - e(r)]}$ 
  
\subsection{Feedback - droplet formation reduces vapour pressure}

\subsection{Aerosol Distributions}

\section{Model}
\subsection{Overview}
\bf{put in a diagram}

\subsection{Initialization}

\subsection{integration}

\end{document}

 
